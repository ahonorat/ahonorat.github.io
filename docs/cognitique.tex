\documentclass[10pt]{article}

\usepackage[utf8]{inputenc}
\usepackage[T1]{fontenc}

\usepackage[french]{babel}

\usepackage{hyperref}
\hypersetup{colorlinks=true}


\title{Introduction très brève à la cognitique pour l'ergonomie des IHM informatiques}
\author{Alexandre Honorat}
\date{Mars 2019}

\begin{document}

\maketitle

Ce document est un résumé partiel du cours d'introduction à la cognitique donné par
Benoît Le Blanc à l'ENSEIRB-MATMECA en 2013.
Cette œuvre est mise à disposition sous licence
Attribution - Pas d’Utilisation Commerciale 3.0 France.
Pour voir une copie de cette licence, visitez \url{http://creativecommons.org/licenses/by-nc/3.0/fr/}
ou écrivez à Creative Commons, PO Box 1866, Mountain View, CA 94042, USA.


\section{Qu'est-ce que la cognitique ?}

Cognitique = Connaissances + Automatique

Les connaissances sont très dépendantes du contexte, elles peuvent également être implicites (travaux d'Ikujiro Nonaka)
ou communautaires (travaux d'{\' E}tienne Wenger). L'automatique est liée à la biologie de l'être humain, et notamment
à sa mémoire. L'application de la cognitique va de la signalétique des aéroports au game-play des jeux vidéos.

\subsection{Modèle de mémoire d'Endel Tulving}

Les mémoires du type 1. à 4. (inclus) concernent les représentation tandis que le type 5., la mémoire procédurale,
concerne les actions. Ce modèle n'est pas unique, ni figé : il permet seulement de distinguer sommairement
les différentes connaissances que nous possédons. À noter que la mémoire à court terme est très limitée :
on ne retient en moyenne que $7$ éléments, $10$ au maximum. Certaines techniques permettent d'améliorer cela,
notamment en associant une image aux éléments à retenir, c.f.
\href{https://fr.wikipedia.org/wiki/M\%C3\%A9thode_des_loci}{palais de la mémoire}.
Enfin, la mémoire est beaucoup plus pérenne qu'on ne le pense, mais pour la retrouver il faut parfois
se replacer dans le contexte dans lequel le souvenir a pu se former, c.f. article
\href{https://lejournal.cnrs.fr/billets/nos-souvenirs-cest-du-solide}{«Nos souvenirs, c'est du solide !»}.


\begin{enumerate}
\item{Mémoire épisodique}
\item{Mémoire à court terme}
\item{Mémoire sémantique}
\item{Représentations perceptives}
\item{Mémoire procédurale}
\end{enumerate}


\subsection{Aide à la décision}

Un des intérêts de l'ergonomie est de faciliter la prise de décision. Cette prise de décision
peut être décomposée en quatre étapes intermédiaires.

\begin{enumerate}
\item{Perception (acquisition): représentation des connaissances, construction circonstancielle}
\item{Compréhension (classification): activation des connaissances, construction permanente}
\item{Raisonnement (inférence)}
\item{Décision}
\end{enumerate}

Un exemple de mauvaise ergonomie est la présence d'une poignée ou d'une barre sur une porte qu'il faut pousser :
la poignée ou la barre invite implicitement à tirer la porte vers soi, ce qui est l'opposé de l'action souhaitée.
Cet exemple est tiré du livre \underline{The Designe of Everyday Things}, \textit{Donald A. Norman}, MIT Press
(en anglais, ISBN : 978-0-262-52567-1).

Notez que l'aide à la décision passe aussi par l'étude de tout ce qui \emph{n'aide pas} à la décision.
Nous effectuons la plupart de nos décisions de manière très biaisée, c.f. notamment les travaux
d'\href{https://fr.wikipedia.org/wiki/Heuristique_de_jugement}{Amos Tversky et de Daniel Kahneman}.
En marketing, l'aide à la décision est très étudiée, mais pas forcément pour le bien du consommateur !
Référence technique à ce sujet : \underline{Nos préférences sous influence}, \textit{Olivier Corneille}, Mardaga
(ouvrage technique, ISBN : 978-2-8047-0031-7).


\section{Interfaces Homme Machine (IHM)}


Interface = Fonction + Contexte

Le contexte dépend des cas d'usages, tandis que la fonction sont les tâches effectuées
par la machine. L'utilisateur donne les commandes à la machine.

\subsection{Conception des IHM}

Pour concevoir une IHM, il faut avant tout connaître les automatismes que l'utilisateur a déjà,
afin si possible de les réutiliser pour le guider, ou en tous cas, pour ne pas le perdre.

\paragraph{Somatisation des informations :}
De très nombreuses informations sont somatisées par haibtude. Par exemple l'ordre des chiffres sur un digicode
d'immeuble ou de distributeur de billets est toujours le même sans que l'on s'en rende compte.
Autre exemple, tous les plans de métro (excepté peut-être celui de la ligne B du métro de Rennes !)
contiennent des lignes droites et non courbes, alors que cela ne correspond pas à la réalité ; mais c'est
beaucoup plus facile à lire\footnote{À propos de la représentation des données, voir les travaux d'\href{https://en.wikipedia.org/wiki/Edward_Tufte}{Edward Tufte}.}.

\paragraph{Quelques principes généraux de conception :}
Mettre peu de texte, user des couleurs avec précautions (attention aux choix esthétiques et aux daltoniens),
se focaliser sur le contexte d'usage, penser à l'évolution, limiter les informations non pertinentes
(la mémoire à court terme ne peut contenir que $7$ éléments !), diminuer le temps d'exécution de la tâche
(notamment pour éviter les erreurs d'innatention). Enfin, le principe anglo-saxon
le plus connu pour le design : «~Less is best!~».


\subsection{Vocabulaire}

\paragraph{Utilisateur $\neq$ Usager :}
L'utilisateur est lié à l'outil et à l'action tandis que l'usager est lié au service et à l'environnement.
Par exemple le conducteur d'un bus est un utilisateur, mais les passagers sont des usagers.

\paragraph{Utilité $\neq$ Utilisabilité :}
L'utilité est la capacité de l'outil à aider dans une réalisation. L'utilisabilité correspond à la
facilité d'emploi de l'outil. L'utilisabilité s'évalue par rapport à l'apprenabilité (du fonctionnement
de l'outil), ainsi que par rapport à l'efficacité, l'efficience, et la satisfaction (telles que définies
dans la norme ISO 9241).

\section{Principales heuristiques pour les IHM informatiques (mais pas que)}

Ces heuristiques ne sont évidemment pas les seules, et reposent sur l'expérimentation plutôt que sur la théorie.
L'évaluation de la qualité des IHM est une tâche assez difficile. Toutefois pour l'informatique deux mesures simples
peuvent être effectuées : le nombre de clics et la distance parcourue par la souris pour réaliser une tâche.

\subsection{Critères ergonomiques de Dominique Scapin et Christian Bastien}

Ces critères sont centrés sur l'humain (du point de vue
biologique) et s'appliquent sur la tâche et son environnement, c.f. \href{https://hal.inria.fr/inria-00070012/fr/}{article original}.

\begin{enumerate}
\item{Guidage}
\item{Charge de travail}
\item{Contrôle explicit}
\item{Adaptabilité}
\item{Gestion des erreurs}
\item{Homogénéïté et cohérence}
\item{Signifiance des codes}
\item{Compatibilité}
\end{enumerate}

\subsection{Règles d'or de Ben Schneiderman}

Ces règles sont centrées sur l'utilisateur et sur
l'expertise.

\begin{enumerate}
\item{Garder une cohérence}
\item{Autoriser des raccourcies pour les habitués}
\item{Fournir des retours informatifs (surtout pour les actions rares et majeures)}
\item{Concevoir les dialogues autour de buts}
\item{Offrir une gestion simple des erreurs}
\item{Autoriser la réversibilité des actions}
\item{Contribuer au sentiment de contrôle}
\item{Réduire la charge mentale}
\end{enumerate}

\subsection{Heuristiques de Jakob Nielsen}

\begin{enumerate}
\item{L'état du système doit être visible}
\item{Le système doit être le reflet du monde réel}
\item{L'utilisateur doit garder le contrôle et être libre}
\item{Être cohérent et respecter les standards}
\item{Prévenir les erreurs}
\item{Reconnaître plutôt que rappeler}
\item{Flexibilité et efficacité}
\item{Esthètique et minimalisme}
\item{Aider l'utilisateur à reconnaître, diagnostiquer, et réparer les erreurs}
\item{Aide en ligne et documentation}
\end{enumerate}


\end{document}
